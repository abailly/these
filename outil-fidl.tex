
Nous revenons dans ce chapitre \`a des aspects plus pratiques de la
mise en  \oe uvre des mod\`eles et concepts pr\'esent\'es
pr\'ec\'edemment dans le cadre d'un processus de d\'eveloppement. 
L'objectif de ce chapitre est de d\'ecrire d'une part comment sont produits et
manipul\'es les mod\`eles \textsf{FIDL} d'architecture de
composants et d'autre part comment la g\'en\'eration et
l'ex\'ecution des tests \`a partir des mod\`eles s'ins\`ere dans
le processus de d\'eveloppement. 

\section{Processus de test}

Nous d\'etaillons dans cette section le processus concret de test de
composants \`a partir de mod\`eles \textsf{FIDL}.

\subsection{Plateforme de test}

\`A partir d'une description quelconque d'un mod\`ele FIDL, par
exemple en utilisant la syntaxe \textsf{IDL3} \'etendue
pr\'esent\'ee au chapitre \ref{chap-fidl}, on construit une
repr\'esentation m\'emoire des diff\'erents objets
sp\'ecifi\'es. Ces objets sont r\'epartis en quatre cat\'egories distinctes :
\begin{itemize}
  \item les types de donn\'ees primitifs et construits, y compris les types
    d'\'ev\'enements et les types d'exception ;
  \item les entit\'es arcitecturales, interfaces, composants et
    assemblages ; 
  \item les descriptions de comportements autrement dit les automates
    \textsf{FIDL} ;
  \item les fonctions et pr\'edicats utilis\'es dans les contraintes
  sur les automates \textsf{FIDL}.
\end{itemize}

Les types de donn\'ees, les fonctions et pr\'edicats sont
compil\'es dans le langage d'implantation choisi pour la partie
fonctionnelle du mod\`ele \textsf{FIDL}. En l'\'etat actuel de la
plateforme, ces objets permettent de produire un ensemble de types de
donn\'ees et de fonctions du langage Jaskell qui sont ensuite
compil\'es vers du code-octet \textsf{Java}. 

En fonction du choix de la plateforme technique d'ex\'ecution des
tests, c'est \`a dire du choix du mode de d\'eploiement et
d'implantation des composants, un certain nombre de souches sont
g\'en\'er\'ees \`a partir des interfaces ou extraites de
l'implantation. Dans le cas des composants CCM et POJO, les
projections des interfaces fournies et requises par le composant sont
normalement incluses dans l'ensemble des classes d'implantation et il
n'est donc pas n\'ecessaire de les g\'en\'erer. 

On choisi ensuite la sp\'ecification \`a tester et on d\'efinit le
mode de d\'eploiement du CUT. \`A chaque plateforme technologique
est associ\'e un d\'eployeur sp\'ecifique qui a pour fonction :
\begin{itemize}
  \item de construire un contexte d'ex\'ecution permettant au CUT de
  s'ex\'ecuter normalement. Dans le cas de composants CCM ou de
  composants de type applications Web ou EJB, ce contexte
  d'ex\'ecution est un container du type appropri\'e. Dans le cas de
  composants POJO, ce contexte est simplement un flot d'ex\'ecution
  Java repr\'esentant l'application dont peut faire partie le
  composant ;
\item de r\'ealiser et de maintenir les connexions correspondant \`a
  chaque port du composant. Chaque connexion doit assurer la
  transformation des messages dans la technologie appropri\'ee pour
  le port consid\'er\'e et inversement, ce que l'on nomme
  \emph{marshalling}. \`A noter qu'il n'est pas n\'ecessaire qu'un
  composant donn\'e se limite \`a un mode de connexion. Par exemple,
  un composant servlet peu avoir un port fourni communiquant par
  protocole HTTP et un port requis communiquant par RPC. 
\end{itemize}
On notera que le connecteur doit aussi assurer la transformation des
donn\'ees transport\'ees par les messages depuis/vers les types
manipul\'es par les fonctions et pr\'edicats dans les automates
\textsf{FIDL}.

Une fois le composant \`a tester pr\^et \`a s'ex\'ecuter, on
d\'etermine les param\`etres de la session de test :
\begin{itemize}
  \item l'automate actif pour la session d\'etermine la strat\'egie
    de test pour cette session. Cet automate peut \^etre soit
    l'automate de la sp\'ecification d'un port, soit un automate de la
    sp\'ecification du composant ; 
  \item les param\`etres de contr\^ole de la fonction d'\'evaluation
    : valuation des diff\'erents crit\`eres de s\'election et
    objectif \`a atteindre pour la session ;
  \item les g\'en\'erateurs pour les variables apparaissant dans les
  messages d'entr\'ee du composant peuvent \^etre param\'etr\'es,
  soit avec un enesemble fini de valeurs \`a tester, soit avec une
  fonction de g\'en\'eration. Ce point est un des plus d\'elicats
  car du choix des valeurs peut d\'ependre l'ex\'ecutabilit\'e de
  certains chemins dans l'automate \textsf{FIDL}.
\end{itemize}

La session peut alors d\'emarrer : le contr\^oleur de test
g\'en\'eral met en \oe uvre les algorithmes d\'etaill\'es dans le
chapitre pr\'ec\'edent pour choisir les chemins d'ex\'ecution de la
session de test et v\'erifie que les messages provenent du CUT sont
bien conformes \`a ceux attendus. En cas de besoin, le CUT peut
\^etre reinitialiser et donc red\'eployer.

\subsection{Exemple de mise en \oe uvre}



\section{Mod\`eles et processus de d\'eveloppement}

La premi\`ere \'etape consiste \`a disposer d'un mod\`ele
\textsf{FIDL} de l'application. Ce mod\`ele peut provenir de
diff\'erentes sources sous r\'eserve de disposer d'une extension
sp\'ecifique au langage utilis\'e pour d\'ecrire le mod\`ele
FIDL.  Un mod\`ele FIDL est suppos\'e \^etre exprim\'e sous la
forme d'une instance du m\'eta-mod\`ele FIDL. La transformation \`a
partir d'un autre mod\`ele, par exemple, UML, EdoC, Fractal ou autre,
est une \'etape pr\'ealable. Cette instance peut-\^etre sous
diff\'erentes formes dont les plus courantes sont :
\begin{itemize}
  \item le format \textsf{XMI} qui une repr\'esentation XML d'un
  mod\`ele ou m\'eta-mod\`ele \textsf{UML} ;
\item le format \textsf{IDL3} annot\'e, dont le corps respecte la
  grammaire usuelle des descriptions de composants IDL3 et dont les
  commentaires peuvent contenir des descriptions de comportement sous
  la forme d'expressions \textsf{FIDL}. 
\end{itemize}

Ce mod\`ele constitue l'entr\'ee d'un certain nombre de modules :
\begin{itemize}
  \item le contr\^oleur de mod\`ele qui v\'erifie la coh\'erence
  s\'emantique des mod\`eles, c'est \`a dire les propri\'et\'es
  de compositionnalit\'e d\'efinies au chapitre
  \ref{cha:composition} : ind\'ependance des facettes et respect des
  ports. Le contr\^oleur de mod\`ele v\'erifie aussi la
  validit\'es des relations de sous-typage entre interfaces et entre
  composants ;
\item le g\'en\'erateur de test qui produit des cas de tests
  statiques abstraits \`a partir des descriptions comportementales
  d'un composant et d'objectifs de couverture ;
\item le g\'en\'erateur de testeur qui produit un testeur dynamique,
  c'est \`a dire un programme ex\'ecutable testant dynamiquement
  selon les algorithmes pr\'esent\'es dans le chapitre
  pr\'ec\'edent un ou plusieurs composants ;
\item un d\'eployeur qui, \`a partir d'un mod\`ele et d'une
  implantation dans une ou plusieurs plate-formes sp\'ecifiques est
  capable de lier un testeur et  un composant r\'eel. 
\end{itemize}

\subsection{D\'eploiement \& Ex\'ecution}

Le d\'eploiement d'un composant r\'eel est une op\'eration
d\'ependant du medium de communication et de la structure du
composant lui-m\^eme. Par d\'eploiement, on
entend la construction d'un environnement permettant \`a un testeur
et \`a un composant de communiquer. Le r\^ole du d\'eployeur est
\`a l'ex\'ecution de traduire les messages entre le formalisme
\textsf{FIDL} et le protocole effectif utilis\'e par le composant
test\'e.

Un d\'eployeur est configur\'e d'une part par le mod\`ele
d'ex\'ecution du composant, d'autre part par un mod\`ele de
communication. Le mod\`ele d'ex\'ecution d\'efinit comment est
suppos\'e s'ex\'ecuter le composant r\'eel. Dans le cas le plus
simple, il n'y a rien \`a faire : le composant est d\'ej\`a
d\'eploy\'e par ailleurs et seule la communication doit \^etre
prise en charge. Dans d'autre cas (CCM, EJB, WebApps), un contexte
d'ex\'ecution simul\'e totalement ou partiellement doit \^etre mis
en place. 

Le mod\`ele de communication d\'efinit concr\`etement comment
transformer les messages entre le composant et le testeur. En fonction
de la configuration choisie, le d\'eployeur met en place un canal de
communication par port du composant \`a tester, c'est \`a dire par
port identifi\'e dans le mod\`ele \textsf{FIDL}. Il n'est pas
n\'ecessaire que tous les ports communiquent selon un m\^eme
protocole. Un composant peut par exemple communiquer sous la forme
d'appels de m\'ethodes classiques en  \textsf{Java} et  par ailleurs
sous la forme de flux dans des fichiers selon un mod\`ele
\textsf{UNIX}. Une autre configuration courante est celle d'un
composant Web communiquant avec un client dans un protocole
\textsf{HTTP} et avec un serveur en appels de m\'ethodes. 

Les protocoles pris en charge par le prototype sont :
\begin{itemize}
  \item l'appel de m\'ethode local (dans une m\^eme \textsf{JVM}) ;
  \item l'appel de m\'ethode distant par \textsf{RMI},
    \textsf{XML-RPC} ou \textsf{SOAP} ;
  \item l'\'echanges de donn\'ees dans des formulaires \textsf{HTTP},
    soit directement par g\'en\'eration de requ\^etes/r\'eponses
    HTTP 1.1, soit par construction de requ\^etes/r\'eponses dans un
    contexte de \emph{Servlets}.
\end{itemize}

Les mod\`eles d'ex\'ecution pris en charge sont :
\begin{itemize}
  \item composants \textsf{POJO} ;
  \item composants \textsf{OpenCCM} ;
  \item composants \textsf{J2EE} : \emph{Servlets} et \textsf{EJB}.
\end{itemize}

\subsection{V\'erifications}

L'outil de v\'erification effectue donc un certain nombre de
contr\^oles sur le ou les mod\`eles FIDL qui lui sont fournis. Les
v\'erification syntaxiques et de bonne formation des expressions sont
r\'ealis\'ees \`a la vol\'ee par l'analyseur syntaxique \`a
partir d'un descripteur \textsf{IDL3}. Un mod\`ele \textsf{FIDL} en
m\'emoire est suppos\'e correctement form\'e, au sens de
\ref{def:automate-bien-forme}. 

Dans l'\'etat actuel du prototype, toutes les v\'erifications sont
effectu\'ees sur une version simplifi\'ee du langage des composants
et des interfaces. Ce langage simplifi\'e ignore les contraintes et
distingue uniquement les valeurs litt\'erales et les variables,
d\'enot\'ees par leur type. La premi\`ere \'etape de
v\'erification consiste en la construction des langages des
diff\'erents objets, selon les d\'efinitions du chapitre
\ref{cha:composition} : langages d'interfaces, langages de composants,
langages d'assemblages. 

Les v\'erifications r\'ealis\'ees sont donc :
\begin{itemize}
  \item pour les interfaces dans une relation d'h\'eritage, la
    v\'erification de l'h\'eritage comportemental au sens de
    \ref{sec:heritage-comportemental} ;
  \item pour les composants et assemblages :
    \begin{itemize}
      \item la v\'erification du respect des facettes et r\'eceptacles
        ;
      \item la v\'erification de l'ind\'ependance des facettes et des
        puits (sur les alphabets abstraits) ;
      \item \'eventuellement le respect de l'h\'eritage comportemental
        lorsqu'une relation d'h\'eritage explicite est donn\'ee.
    \end{itemize}
\end{itemize}

La relation contractuelle peut aussi \^etre v\'erifi\'ee
explicitement entre deux composants ou assemblages. 

\subsection{Test}

Les tests \`a partir de mod\`eles \textsf{FIDL} peuvent \^etre
r\'ealis\'es selon deux modalit\'es :
\begin{itemize}
  \item traditionnelle : un certain nombre de cas de
  test sont g\'en\'er\'es, puis ensuite ex\'ecut\'es en tant que
  cas de test \textsf{JUnit} ;
\item dynamique : un testeur implantant l'interface
  \textsf{TestRunner} de \textsf{JUnit} ex\'ecute des tests de
  mani\`ere adaptative en fonction des r\'eponses re\c{c}ues depuis
  l'\textsf{IUT} et selon les strat\'egies d\'efinies dans le
  chapitre \ref{cha:selection-des-cas}.
\end{itemize}

Dans ce second cas de figure, l'int\'egration avec JUnit se fait en
consid\'erant que toute op\'eration de \texttt{reset} signale une
fin de test unitaire r\'eussi qui est donc indiqu\'ee comme tel au
framework \textsf{JUnit}.

\subsection{Couverture de code}

Le test \`a partir d'un mod\`ele \textsf{FIDL} peut \^etre
\'evidemment coupl\'e avec une analyse de la couverture du code du
composant r\'eel lorsque celui-ci est disponible localement. Cette
information est extr\^emement int\'eressante puisqu'elle permet de
savoir quelle fraction du code d\'evelopp\'e est couverte par la
sp\'ecification mod\'elis\'ee ou de mani\`ere \'equivalente quel
partie de la sp\'ecification est r\'eellement ex\'ecutable dans le
code. 

\subsection{Granularit\'e du test}

Le principal int\'er\^et du mod\`ele d'architectures que nous avons
propos\'e est de permettre de tester \`a partir du m\^eme outil et
de fa\c{c}on syst\'ematique les fonctionnalit\'es d'un ensemble de
composants produisant un syst\`eme par assemblage. Selon les choix
fait dans le processus de d\'eveloppement et le degr\'e de
pr\'ecision de la mod\'elisation, les tests pourront \^etre
r\'ealis\'es \`a diff\'erents niveaux. 

Dans le cas le plus simple, seul un sous-syst\`eme complet, par exemple un
cas d'utilisation dans l'exemple de la m\'ethodolgie pr\'esent\'ee
au chapitre \ref{cha:proc-de-devol}, sera mod\'elis\'e de mani\`ere
formelle en utilisant des sp\'ecifications \textsf{FIDL}. \`A partir
de cette mod\`elisation on produira alors un  testeur syst\`eme
dont les connexions seront instanci\'ees avec les protocoles les plus
externes de l'application consid\'er\'ee. 

Dans le cas id\'eal, chaque composant de chaque couche de
l'application sera mod\'elis\'e et sp\'ecifi\'e sous la forme d'un
composant FIDL, l'ensemble \'etant organis\'e sou forme d'assemblage
hi\'erarchique. Les propri\'et\'es que nous avons d\'emontr\'ees
dans les chapitres pr\'ecedents permmettront d'all\'eger l'effort de
test d'int\'egration : la v\'erification du respect des contrats par
les composants garantit par la suite la correction de l'assemblage.

\section{Exemple}



\section{Outillage}

La majeure partie des mod\`eles et algorithmes d\'efinis dans les
chapitres pr\'ec\'edent a \'et\'e implant\'ee sous la forme de
biblioth\`eques et de programme \textsf{Java} destin\'es \`a
s'int\'egrer dans un processus de d\'eveloppement et
d'int\'egration continue. Nous pr\'esentons ici sommairement chacun
de ces composants. 



Strat\'egie g\'en\'erale : pas de plateforme unique (eg. Eclipse)
faisant tout, processus souple et agile, ind\'ependant technologie :
mod\`ele UniX : un ensemble de petits outils qui font chacun une
chose et le font bien. ID : extrapolation du mod\`ele de flux texte
Unix au flux arborescents/graphes (textutils), production de scripts
de transformation/manipulation de mod\`eles (cf. travaux autour de
QVT, transformation de mod\`eles, MDA, MDE, ...)

production de mod\`eles testables : m\'eta-mod\`eles bas\'es sur
une architecture de composants, extractions de mod\`eles \`a partir
d'existant (approche Hood), transformations de mod\`eles existants :
mod\`eles de classes UML avec st\'er\'eotypes, mod\`eles de
composants, mod\`eles AADL, ...format pivot = IDL3/FIDL

partie dynamique : lien entre diff\'erents mod\`eles \`a base de
syst\`emes de transitions d'\'etats finis : mod\`ele FIDL
repr\'esente une approximation pessimiste et v\'erifiable du
comportement r\'eel.  Lien avec IOLTS (LOTOS) et EFSM
(Statecharts). Voir liens avec OCL : description des contraintes. 

liaison avec le code : PSM : couche de projection entre les mod\`eles
de composants et les mod\`eles concrets. rester le + longtemps
possible dans les couches abstraites. 

Mod\`ele de composant
utilis\'e \`a la fois pour v\'erification et pour d\'eploiement :
production de descripteurs automatiques, g\'en\'eration de code
d'interfaces/stubs/squelettes/souches/...

V\'erification existant vs. projection (cf chapitre 1, conclusion) :
ex. Macker, v\'erification de contraintes architecturales \`a partir
des r\`egles de projection : permet d'utiliser le m\^eme formalisme
et les m\^emes jeu de r\`egles pour produire du code et pour
v\'erifier du code existant. 

Int\'egration dans le processus de d\'eveloppement : versionning de
mod\`ele/code g\'en\'er\'e et enrichi. Pb: distinguer les parties
g\'en\'er\'ees des parties \'ecrites, maintenir la coh\'erence
entre les mod\`eles et le code source. Utilisation de balises, diff,
syst\`eme standard de gestion de version, r\'econciliation (merge) :
\'evolution parall\`ele des mod\`eles et du code. 

R\'ef\'erentiel de mod\`eles : modfact, mdr, emfcore. Outil central
pour le d\'eveloppement.

Ex\'ecution des tests : g\'en\'eration de testeurs sp\'ecifiques
au d\'eploiement. Production de connecteurs ad-hoc en fonction de la
plateforme cible : les tests sont g\'en\'eriquement ex\'ecut\'es
et transform\'es \`a la vol\'ee en fonction de la cible. (exemple :
OpenCCM, Java). Connecteurs entre deux techno : Java <-> Http ? Java
<-> OpenCCM, OpenCCM <-> .Net, ....

Couverture / Fiabilit\'e des tests : d\'efinition d'une strat\'egie
adaptative en fonction des risques, possibilit\'e d'atteindre 100\%
dans certains cas, mod\`ele g\'en\'erique de couverture applicable
\`a n'importe quel syst\`eme de transition. 

Strat\'egie de test d'architecture : diminuer l'int\'er\^et du test
d'int\'egration en accroissant la qualit\'e et la fiabilit\'e des
tests unitaires (test de conformit\'e, test d'ind\'ependance, test
de r\'esilience). Utiliser diff\'erents niveaux de description
d'architecture : strat\'egie risque/rentabilit\'e, optimiser le
nombre de tests et les temps d'ex\'ecution sur les \'el\'ements les
plus strat\'egiques (voir en fonction de la complexit\'e des
automates, mesure du nombre cyclomatique : choisir d'abord les
sp\'ecifications les plus complexes ou leur allouer plus de temps
d'ex\'ecution que les autres voir mesure de la fiabilit\'e en
fonction du temps CPU de Musa\cite{musa-theory-sr}

Objectif g\'en\'eral : accroissement  de la fiabilit\'e => mesure
\`a partir des anomalies remont\'ees en TMA (hors demande
d'\'evolutions) : n\'ecessite de faire une \'etude sur l'impact du
test architectural sur la fiabilit\'e g\'en\'erale : analyse des
erreurs d\'etect\'ees par le test avant recette et des erreurs
r\'ev\'el\'ees en recette (historique des erreurs :
traceabilit\'e, cas de test).

\section{M\'ethodes \& Outils pour les architectures de composants}

\section{Pratique du test de composants \textsf{FIDL}}

\subsection{La plateforme de test \textsf{FIDL}}

Analyse de mod\`ele FIDL : v\'erifications  de base (syntaxe,
s\'emantique : projections sur alphabets abstraits, ind\'ependance,
....)

G\'en\'eration de cas de test


%%% Local Variables: 
%%% mode: latex
%%% TeX-master: "these"
%%% TeX-master: "these"
%%% End: 
  