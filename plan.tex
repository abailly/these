\documentclass[french,a4paper,leqno,twoside]{book}
\usepackage{a4wide}
\usepackage{amsmath}
\usepackage{amssymb}
\usepackage{babel}
\usepackage[latin1]{inputenc}


\title{Plan de th\`ese}
% $Id$
\begin{document}
\nopagebreak
\maketitle
% \tableofcontents

\chapter{Introduction}
\section{Contexte}
\section{Probl\'emes}
\section{R\'esum\'e}

\part{Composants}

\chapter{Mod\`eles pour les composants}

\section{Mod\`ele formels}

\subsection{Langages de descriptions d'architecture}
 (Wright, Rapide)
\subsection{Co-alg\`ebres}
 (REO, Composants g\'en\'eriques)
\subsection{Calculs de pracessus}
 (kell-calcul, pi-calcul, ambients)
\subsection{M\'eta-mod\`eles UML}
 (eDoc, UML 2.0)

\section{Implantation}

Syst\`emes, Langages, Plates-formes et intergiciels

\subsection{Modules, biblioth\`eques}
oldies but goldies : DLL (DLL nightmare), shared object, compilation
s\'epar\'ee, fichiers en-t\^ete (.h)
\subsection{Corba 3.0}
\subsection{J2EE}
\subsection{.Net}
\subsection{Langages d\'edi\'es}
ArchJava
\subsection{Conteneurs logiques}
 (Spring, Pico)


\chapter{Mod\`ele de composant FIDL}

\section{Mod\`ele}

 - Interfaces 
 - \'Ev\'enements 
 - Composants
 - Composition
 - Dynamicit\'e
 - M\'eta-mod\`ele UML
 
\section{Le langage FIDL}

\subsection{La notation IDL3}
\subsection{Syntaxe FIDL}
\begin{itemize}
  \item Structure 
  \item Comportement
\end{itemize}

\section{S\'emantique - Traces} 
 
\begin{itemize}
  \item D\'efinitions, notations
  \item Alphabets
  \item Traces d'une interface
  \item Traces d'un composant
  \item Connexions
  \item Traces d'un composite
  \item Syst\`eme
\end{itemize}

\chapter{Automates FIDL}

\begin{itemize}
  \item d\'efinition du formalisme des automates FIDL, lien avec les
    IOLTS, FSM et assimil\'es
  \item Construction
  \item d\'efinition des fonctions utilis\'ees dans les donn\'ees
    des messages
  \item preuve de la rationalit\'e des automates FIDL d\'epli\'es
    (avec les valeurs instanci\'ees)
  \item d\'efinition de la reconnaissance de langages de messages par
    les automates FIDL, simples, puis synchronis\'ees
  \item preuve de l'\'equivalence entre le langage d'une produit de
    synchronisation et la synchronisation d'automates
  \item d\'efinition des langages des diff\'erents \'el\'ements
    d'un syst\`eme : interface, port, composant, composants
    connect\'es
  \item d\'efinition du probl\`eme de r\'esolution de contraintes
    pour la g\'en\'eration de valeurs
\end{itemize}

\chapter{Composition \& H\'eritage}

\section{Composition}

\section{H\'eritage \& Sous-typage comportemental}

\part{Test de composants} 

\chapter{\'Etat de l'art : le test de logiciels}

\begin{itemize}
  \item Test fonctionnel/structures
  \item Test bo\^{\i}te noire/bo\^{\i}te de verre
  \item Test unitaire/syst\`eme
  \item Test statistique/al\'eatoire
\end{itemize}

\chapter{Test de composants}

\section{Le test de conformit\'e}
\begin{itemize}
  \item d\'efinition du contexte de test : notion de testeur de port,
    de testeur de composant, de langage observ\'e et observable, de
    conformit\'e du composant test\'e par rapport au langage du
    testeur.
  \item d\'efinition de la relation de conformit\'e *associ\'ee* au
    ou *v\'erifi\'ee* par le testeur de composant (cf. MIOTS,
    inclusion de traces)
  \item d\'efinition des **observateurs** de facettes, de ports, de
    composants par compl\'etion des sp\'ecifications associ\'ees
    aux diff\'erents ports
  \item d\'efinition d'un testeur de conformit\'e comme un composant
    dont le comportement est exactement celui r\'esultant de la
    composition des observateurs
  \item d\'efinition de l'algorithme de test pour ports multiples
\end{itemize} 

\section{Le test de r\'esilience}
\begin{itemize}
   \item d\'efinition de la propri\'et\'e de transparence
  \item test de la propri\'et\'e de compositionnalit\'e :
    d\'efinition d'un testeur pour la transparence
  \item d\'efinition de la r\'esilience
  \item d\'efinition d'un testeur pour la r\'esilience
  \item algorithme pour tester la transparence et la r\'esilience
\end{itemize}

\section{S\'election des tests}

\part{Test \& G\'enie logiciel}

\chapter{Processus de d\'eveloppement et composants}

\section{M\'ethodologie orient\'ee composants}
\section{Ing\'enierie des mod\`eles}
\section{\'Etude de cas}

\chapter{M\'ethodes \& outils}
 
\section{Tests unitaires}
\section{Tests fonctionnels et d\'evellopement}
\section{Tests d'int\'egration}
\section{Recette \& Tests d'acceptation}

\chapter{Outil FIDL}
\section{Architecture}
\section{Exemple}
\section{\emph{Test-driven development}}

\chapter{Conclusion}
\section{Contributions}
\section{Perspectives}

\appendix
\chapter{Grammaire FIDL}
\chapter{Jaskell}
\end{document}

%%% Local Variables: 
%%% mode: latex
%%% TeX-master: t
%%% End: 
