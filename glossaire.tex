\paragraph{composant} 
\label{def-composant}
\'El\'ement autonome d'un syst\`eme\ref{def-systeme} susceptible d'\^etre
interconnect\'e avec d'autres composants au moyen de ports.

\paragraph{systeme}
\label{def-systeme}
Composite\ref{def-composite} ex\'ecutable sans d\'ependances externes.

\paragraph{composite}
\label{def-composite}
Assemblage de composants et de composites poss\'edant des
d\'ependances externes et non ex\'ecutable de mani\`ere autonome. 

\paragraph{port}
Point de connexion d'un composant vers un autre composant.

\paragraph{connexion}
Lien entre deux ports de deux composants cr\'e\'e lors de leur d\'eploiement et de leur
ex\'ecution. 

\paragraph{interface}
Type des ports synchrones des composants.

\paragraph{facette}
Offre de service d'un composant au travers d'un port synchrone typ\'e
par une interface.

\paragraph{r\'eceptacle}
D\'ependance de service d'un composant au travers d'un port synchrone
typ\'e par une interface.

\paragraph{puits (d'\'ev\'enements)}
Port asynchrone d'un composant typ\'e par un \texttt{eventtype}.

\paragraph{source (d'\'ev\'enements)}
Port asynchrone d'un composant typ\'e par un \texttt{eventtype}.

\paragraph{panne}
D\'eviation du comportement d'un composant test\'e par rapport \`a
sa sp\'ecification (\emph{failure}).

\paragraph{d\'efaut}
Cause d'une panne (\emph{fault}). Une panne est d\'etect\'ee par le
test et un d\'efaut est corrig\'e par le d\'everminage.

\paragraph{erreur}
Action humaine ayant entra\^{\i}n\'e un d\'efaut donc une panne.


%%% Local Variables: 
%%% mode: latex
%%% TeX-master: "~/recherche/text/these"
%%% End: 
